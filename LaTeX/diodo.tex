\documentclass{article}[a4paper, oneside, 11pt]
\usepackage[T1]{fontenc}
\usepackage[utf8]{inputenc}
\usepackage{calc}
\usepackage{amsmath, amssymb, amsthm, thmtools, amsfonts}
\usepackage{mathtools}
\usepackage[nochapters,pdfspacing]{classicthesis}
%\usepackage{hyperref}% clashes with classicthesis
\usepackage{cleveref}
\usepackage[siunitx]{circuitikz}
\usepackage{booktabs}
\usepackage{graphicx}
\usepackage{caption}
\usepackage{geometry}
\usepackage{float}
\usepackage{mdframed}
\usepackage{xcolor}
\usepackage{siunitx}
\usepackage[italian]{babel}
\usepackage{pgfplots}
\usepackage{titling}
\usepackage{listings}
\usepackage{lmodern}
\usepackage{url}
\usepgfplotslibrary{external} 
\tikzexternalize

\pgfplotsset{compat=1.15}
\lstset{
basicstyle=\ttfamily,
columns=fullflexible,
keepspaces=true,
}
\mdfsetup{linewidth=0.6pt}
\graphicspath{{./figs/}}
\makeatletter
\def\input@path{{./figs/}}
%or: \def\input@path{{/path/to/folder/}{/path/to/other/folder/}}
\makeatother

\input{./math}

\geometry{a4paper, left=25mm, right=25mm, top=30mm, bottom=30mm}
\title{Modellizzazione della resistenza di diodi a giunzione PN per alte correnti di lavoro}
\author{L.~Ciucci(\thanks{Dipartimento di Fisica E.~Fermi, Universit\`a di Pisa - Pisa, Italy} )~\and S.~Bruzzesi(\protect\footnotemark[1] )~\and M.~Romagnoli(\protect\footnotemark[1] )~\and M.~Alighieri(\protect\footnotemark[1] )~\and B.~Tomelleri(\protect\footnotemark[1] )}
\date{2020/11/01}

\begin{document}
\maketitle

\begin{mdframed}
\textbf{Riassunto:} --- Studiamo il comportamento di diodi in silicio PN,
esplorando la propria curva caratteristica $V - I$ al di fuori dei regimi di
lavoro ordinari, tramite campionamenti digitali dei segnali ai capi del
componente. Discutiamo la presenza di una componente resistiva del diodo e ne
misuriamo l'entità, al fine di arrivare ad un modello teorico in grado
giustificare eventuali deviazioni dal modello di Shockley, verso una risposta
-ohmica- dovuta alla resistenza della giunzione PN al passaggio di correnti.\\\\
PACS 01.40.-d – Education.\\
PACS 01.50.Pa – Laboratory experiments and apparatus.
\end{mdframed}

\section{Introduzione}
L'alta resistività intrinseca al silicio, di cui è composta la giunzione
bipolare, comporta la presenza di una sua componente resistiva: questa 
risulta sempre meno trascurabile agli effetti del passaggio di corrente
attraverso il diodo, all'aumentare della tensione ai suoi capi e della sua 
temperatura. Per poter modellare la componente -ohmica- di un diodo
percorso da correnti alte si propone un modello semplice di resistenza
parassita in serie, in grado di descriverne gli effetti, verificandone 
sperimentalmente la validità e i limiti. 
\section{Metodo e apparato sperimentale}
Dovendo lavorare con correnti relativamente alte per il circuito sotto studio,
al fine di minimizzare effetti termici-dissipativi secondari ed evitare danni
all'apparato, si imprimono sui componenti correnti pulsate o di durate molto
brevi, limitando dunque il trasferimento di energia sui componenti.
\subsection{Acquisizione dati}
Si è fatto uso della piattaforma di sviluppo \verb+Teensy 3.2+\cite{teensy} per
il campionamento di segnali, essendo non solo più veloce e capiente in memoria
rispetto ad Arduino, ma oltretutto dotato di due ADC, entrambi con risoluzione
maggiore nel campionamento analogico, a 12 bit (reali). In particolare, 
\verb+Teensy+ è in grado di effettuare una lettura differenziale su due ingressi
analogici e sincronizzata tra i due ADC. Questo ci permette di misurare
pressoché -simultaneamente- la differenza di potenziale e intensità di corrente
percepite dal diodo, dunque la sua curva caratteristica. Per prima cosa
si sono misurati i rispettivi fattori di conversione $\xi_{\rm V}$ [V/digit] e
$\xi_I$ [A/digit] tramite un fit lineare sui campionamenti di tensione e
corrente continua generati da un trasformatore di d.d.p $V_0 \approx 4.95$ V.
Come illustrato sotto:
\input{./figs/thev.tex}
\subsection{Schema circuitale del sistema}
%\input{simpletikz.tex}
\begin{figure}[!htb]
	\centering 
 		\includegraphics[scale=2.2]{./simple.pdf}
 	\caption{Versione semplice del circuito \label{sch:smpl}}
\end{figure}
\begin{figure}[!htb]
	\centering 
 		\includegraphics[scale=1.3]{./gestione.pdf}
 	\caption{Circuito globale per la gestione del diodo \label{sch:gest}}
\end{figure}
\begin{figure}[!htb]
	\centering 
 		\includegraphics[scale=2.2]{./measure.pdf}
 	\caption{Schema circuitale del sistema di lettura (\texttt{Teensy})
	\label{sch:rdng}}
\end{figure}
Come ultima cautela preliminare, per minimizzare le influenze esterne sulla
misura della componente resistiva del diodo, si sono misurate le resistenze dei
collegamenti del circuito, in quanto i cavi reali e gli ingressi del diodo
possono avere resistenze non trascurabili rispetto a quella opposta dal diodo
in regime di conduzione, nell'ordine di qualche ohm.

Tramite il semplice circuito con interruttore in Figura \ref{sch:smpl} riusciamo
a ricostruire sperimentalmente la curva caratteristica V-$I$ del diodo: si
lascia scaricare repentinamente il condensatore sulla serie diodo-resistenza
chiudendo l'interruttore, dunque dai canali di un oscilloscopio si misurano le
le cadute di tensione ai capi del diodo e della resistenza $R_P$, da cui si
ricava l'intensità di corrente che scorre su entrambi i componenti tramite la
legge di Ohm.\[
I_d = \frac{\Delta V_{\mathrm{mis}}}{R_P}
.\] 

Vista la limitata attendibilità del circuito con interruttore azionato
manualmente si è costruito una seconda versione del circuito (Fig.
\ref{sch:gest}), in cui figurano: 
\begin{description}
	\item [Un sottocircuito] di switch	
	\item [Un condensatore] di capacità maggiore di un paio di ordini di 
	grandezza.
	\item [Teensy] La piattaforma di sviluppo impiegata per caricare
	(e scaricare) il condensatore a diverse tensioni e per la misura delle
	tensioni ai capi del diodo e della resistenza.
\end{description}

Si sviluppano due casi principali, dipendenti sostanzialmente dal valore della
resistenza $R_P$ posta in serie al diodo:\\

Se lasciamo caricare gradualmente il condensatore, essendo $C$ decisamente
maggiore rispetto ai condensatori precedenti si ha un tempo di carica
$\tau = RC$ abbastanza prolungato, in cui possiamo campionare contemporaneamente
tensione e corrente ai capi del diodo per correnti modeste, nel regime in cui
il diodo è "interdetto" ed oppone resistenza al flusso di carica.

Viceversa, nel regime in cui si applichi al diodo una d.d.p. ben al disopra di
$V_{\rm thr} \approx 0.6 $V, che siamo liberi di esplorare variando la tensione
di carica di C con il sottocircuito destro, il diodo idealmente lascia passare
tutta la corrente impressa sulla giunzione. Allora per caratterizzare la
risposta del diodo senza cambiarne drasticamente le caratteristiche (ad esempio
per eccessiva agitazione termica) si impiega il circuito di switch per
imprimere impulsi di alta corrente e breve durata sulla giunzione PN, di cui
misuriamo la curva caratteristica in risposta con i due ADC di \verb+Teensy+.

Se effettivamente il diodo, oltre ad una certa soglia di d.d.p. inizia ad avere
componente resistiva sempre più pronunciata, allora riportando le nostre
previsioni in carta semilogaritmica ci si aspetterebbe di trovare una retta,
entro il regime in cui è valida l'approssimazione di Shockley, ma oltre a
questo, una regione in cui la curva caratteristica della giunzione ora mostra
una dipendenza apprezzabilmente più lineare/ohmica dell'intensità di corrente
dalla $\Delta$V rispetto a prima, a cui corrisponderebbe il grafico "piatto"
di un logaritmo.  
\section{Risultati e Analisi dati}
\section{Appendice A: Filtraggio Dati}
Il sistema di filtraggio di dati implementato nell'eseguibile si compone di 2
fasi: la prima è l’eliminazione di outliers, la seconda consiste
nell'eliminazione di dati non significativi.
\subsection{Introduzione}
In generale possiamo immaginare che ad ogni misura sia associata una forma
quadratica rappresentata dalla matrice di covarianza della stessa, cioè che
durante la misura si commetta un errore statistico normale noto. Possiamo poi
immaginare che il misurando abbia un’altra matrice di covarianza. Nel
procedimento proposto questo potrebbe essere teoricamente trattato per esteso,
tuttavia non si troverebbe una forma chiusa generale per il problema in
questione. Supponiamo allora che gli errori su $x$ e $y$ siano indipendenti
e che $\sigma_x ^2 \coloneqq \var{x}$ sia nota a priori e che abbia 
distribuzione gaussiana per ogni misura. Questo non è in generale vero, ma 
questa ipotesi può essere trascurata quando la correlazione tra le varianze
delle misure su $x$ e $y$ sono indipendenti dai valori assunti dalle $x$ e $y$
stesse e vi sono “molti” dati entro una deviazione lungo $x$. In questo caso,
infatti, la correlazione viene inclusa nella varianza lungo $y$ e, dal
teorema del limite centrale si vede che la non-normalità delle distribuzioni
è trascurabile.
In ogni modo, nei nostri dati x e y risultano ragionevolmente indipendenti,
quindi l'assunzione dovrebbe essere lecita.
\subsection{Procedimento}
Supponiamo, per ogni punto, che la distribuzione sia gaussiana secondo una
matrice di covarianza diagonale nella base $\{x, y\}$ : allora la densità di
probabilità che un punto che abbia misurato $x$ si trovi a tale ascissa $x_i$
si ricava banalmente integrando lungo $y$ a $x$ fissata:
\[
	\ud P = \frac{1}{\sigma_{x_i} \sqrt{2\pi}}
	e^{-\frac{1}{2}{\frac{(x - x_i)^2}{\sigma_{x_i}^2}}} \ud x
.\] 
Dunque, ripetendo più volte la stessa misura, la probabilità
\[
	P\left( \mid x - x_i \mid  \leq \eps  \right) = \eps G_{x_i} 
.\]
dove \[
	G_{x_i} \coloneqq \frac{1}{\sigma_{x_i} \sqrt{2\pi}}
	e^{-\frac{1}{2}{\frac{(x - x_i)^2}{\sigma_{x_i}^2}}}
.\] 
e $\eps > 0$ è piccolo a piacere. Scegliendo allora solo quelle misure x per
cui vale $\mid x - x_i \mid \leq \eps$, queste saranno in numero intorno a:
\[
	N_i \coloneqq N_{\text{tot}} \frac{G_{x_i}}{\sum_j G_{x_j}} =
		N_{\text{tot}} w_i
.\] 
che definisce implicitamente i pesi $w_i$ con cui si mediano le distribuzioni
di probabilità gaussiane $G_{x_i}$.
Allora, detto:
\[
	G_{y_i} \coloneqq \frac{1}{\sigma_{y_i} \sqrt{2\pi}}
	e^{-\frac{1}{2}{\frac{(\mu_y - y_i)^2}{\sigma_{y_i}^2}}}
.\] 
Per il principio di massima verosimiglianza siamo quindi interessati a
massimizzare la quantità:
\[
	\like = \prod_{i=1}^{n} \prod_{j=1}^{N_i} G_{y_i} = 
	\prod_{i=1}^{n} G_{y_i}^{N_i}
.\] 
Per la monotonia del logaritmo il problema equivale a massimizzare la quantità:
\[
	\ln{\like} = \sum_{i=1}^{n}\ln{G_{y_i}}^{N_{\text{tot}}w_i} = 
	\frac{N_{\text{tot}}} {\sum_{j=1}^{n} G_{x_j}} 
	\sum_{i=1}^{n} G_{x_i} \ln{G_{y_i}}
.\] 
Per cui, a meno di costanti risulta:
\begin{equation}\label{eq: likeconst}
	\ln{\like} - \text{const.} \propto \sum_{i=1}^{n} -G_{x_i} \ln{\sigma_y}
	- \frac{1}{2} G_{x_i} \left( \frac{y_i - \mu_y}{\sigma_y} \right)^2
\end{equation}
Imponendo la condizione di stazionarietà rispetto a $\mu_y$ si ottiene dunque:
\begin{equation}\label{eq: muy}
	\mu_y = \sum_{i=1}^{n} y_i w_i 
\end{equation} 
Una volta sostituito in \eqref{eq: likeconst} quanto appena trovato per $\mu_y$
e imponendo la stessa condizione di stazionarietà rispetto a $\sigma_y$ si ha:
\begin{equation}\label{eq: sigmay}
	\sigma_y^2 = \sum_{i=1}^{n} (y_i - \mu_y)^2 w_i .
\end{equation}
Infine è possibile ricavare la varianza di $\mu_y$ dalla definizione di valore
di aspettazione, riconducendola più volte a integrali di gaussiane di altezze
e ampiezze diverse:
\begin{align}
	\var{\mu_y} &= \sum_{i=1}^{n} \left[w_i^2 \sigma_y^2 + 
	\left(\frac{y_i}{\sum_{j=1}^{n} w_j}  \right)^2 \frac{
	\frac{e^{-\frac{(x-x_i)^2}{3 \sigma_x^2}}} {\sigma_x \sqrt{6 \pi} } +
	\frac{e^{-3\frac{(x-x_i)^2}{4 \sigma_x^2}}} {\sigma_x \sqrt{\pi} } +
	\frac{e^{-\frac{(x-x_i)^2}{\sigma_x^2}}} {\sigma_x \sqrt{2 \pi}}
	} {\sqrt{2 \pi}} \right] =\\ \label{aln: varmuy}
        &= \sum_{i=1}^{n} \left[w_i^2 \sigma_y^2 + 
	\left(\frac{y_i}{\sum_{j=1}^{n} w_j}  \right)^2 \frac{
	e^{-\frac{(x-x_i)^2}{3 \sigma_x^2}} +  
	\sqrt{3}\left( e^{-\frac{(x-x_i)^2}{\sigma_x^2}} -
	\sqrt{2} e^{-3 \frac{(x-x_i)^2}{4 \sigma_x^2}} \right)
	} {2 \sqrt{3}\pi \sigma_x^2} \right]
\end{align}
Riassumendo:\\
Nella \eqref{eq: muy} prendiamo una media dei campionamenti intorno ad un'
ascissa $x$ in esame, pesata sulla distanza che gli $x_i$ hanno da questa. 
Effettivamente quello che stiamo facendo è un stima di densità di kernel,
per cui consideriamo i punti come -sfocati- da un \emph{blur gaussiano};
dove però nel nostro caso riscaliamo la stima in base al valore assunto da $y$.
Lo stesso ragionamento vale per $\sigma_y^2$, si ha una stima della varianza
dei dati la variare di y, pesata sulla distanza dai valori studiati. Dunque
$\mu_y \pm \sigma_y$ ci dà una stima della distribuzione dei nostri dati.
\subsection{$\var{\mu_y}$}
Mentre $\sigma_y$ rappresenta la distribuzione dei dati intorno al valor medio 
$\mu_y$, $\var{\mu_y}$ ci dà un'idea dell’incertezza che attribuiamo alla
miglior stima di y. Questo ci è utile per determinare la convergenza della
stima in funzione dei dati acquisiti.
Infatti: più la densità dei dati è grande rispetto alla deviazione standard
$\sigma_x$, più la stima del valore centrale risulta precisa. Graficamente
la banda di confidenza è più ristretta dove si concentrano i dati, viceversa
tende ad allargarsi dove i dati sono sparsi, a distanze paragonabili a
$\sigma_x$. Dalla seconda somma nell'espressione \eqref{aln: varmuy} segue
allora che la stima del valore centrale è statisticamente significativa solo
quando si media su un intervallo campionato con almeno qualche punto ogni
deviazione $\sigma_x$, altrimenti $\var{\mu_y}$ tende a $+\infty$ come 
$\sim e^{x^2}$in assenza di dati, dove non è possibile stabilire con
precisione il valore di $\mu_y$.
\subsection{Filtro outliers}
La parte più semplice nel filtraggio dati consiste nello scartare tutti quei
punti che distano da $\mu_y$ più di una soglia arbitraria $k$ di deviazioni
standard dalla media $\sigma_x$, ovvero quegli $y_i$ per cui,
indipendentemente dal modello di fit vale $|y_i - \mu_y| \geq k\sigma_y$.
\subsection{Filtro dati non significativi}
Supponiamo di avere 2 set di dati fatti con diverse resistenze, il primo con
una resistenza bassa, il secondo con una alta: Il primo set esplorerà
la regione ad alta corrente, mentre il secondo la regione di basse correnti.
Però il primo insieme conterrà, in generale, anche campionamenti delle zone
basse, ma su queste fornirà dei valori meno significativi: Esponiamo dunque
il criterio sviluppato per ridurre l'influenza di questi punti meno
significativi sulla ricerca dei parametri di best-fit e sulla rappresentazione 
finale dei dati.\\
Per capire se in un certo punto i dati di $A$ sono significativi, calcoliamo
la misura di significatività che abbiamo sviluppato in \eqref{aln: varmuy}$:
\var{\mu_y}$ di $A$ e di $B$. Perciò se $\var{\mu_y}$ di $A$ è maggiore di 
$q\var{\mu_y}$ di $B$, con $q$ arbitrario, questo indica che i dati di $A$
ci stanno dando meno informazioni rispetto a quelli di $B$. A questo punto
è sufficiente controllare tutti i punti scorrendo su tutte le combinazioni
di set per eliminare i dati non significativi, che rendono meno
immediata l'interpretazione il grafico.

\section{Appendice B: Metodo di Fit}
Per determinare i parametri ottimali e le rispettive varianze si \`e
implementato un metodo di fit basato sui minimi quadrati in \verb+Python+
mediante la funzione \emph{curve\_fit} della libreria Scipy\cite{scipy}.
\medskip
\bibliographystyle{IEEEtrandoi}
\bibliography{refs}
\end{document}
