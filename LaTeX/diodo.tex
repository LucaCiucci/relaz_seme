\documentclass{article}[a4paper, oneside, 11pt]
\usepackage[T1]{fontenc}
\usepackage[utf8]{inputenc}
\usepackage{calc}
\usepackage{amsmath, amssymb, amsthm, thmtools, amsfonts}
\usepackage{mathtools}
\usepackage[nochapters,pdfspacing]{classicthesis}
%\usepackage{hyperref}% clashes with classicthesis
\usepackage{cleveref}
\usepackage[siunitx]{circuitikz}
\usepackage{booktabs}
\usepackage{graphicx}
\usepackage{caption}
\usepackage{geometry}
\usepackage{float}
\usepackage{mdframed}
\usepackage{xcolor}
\usepackage{siunitx}
\usepackage[italian]{babel}
\usepackage{pgfplots}
\usepackage{titling}
\usepackage{listings}
\usepackage{lmodern}
\usepackage{url}
\usepgfplotslibrary{external} 
\tikzexternalize

\pgfplotsset{compat=1.15}
\lstset{
language=Python,
basicstyle=\ttfamily,
columns=fullflexible,
keepspaces=true,
}
\mdfsetup{linewidth=0.6pt}
\graphicspath{{./figs/}}
\makeatletter
\def\input@path{{./figs/}}
%or: \def\input@path{{/path/to/folder/}{/path/to/other/folder/}}
\makeatother

\input{./math}

\geometry{a4paper, left=25mm, right=25mm, top=30mm, bottom=30mm}
\title{Modellizzazione della resistenza di diodi a giunzione PN per alte correnti di lavoro}
\author{L.~Ciucci(\thanks{Dipartimento di Fisica E.~Fermi, Universit\`a di Pisa - Pisa, Italy} )~\and S.~Bruzzesi(\protect\footnotemark[1] )~\and M.~Romagnoli(\protect\footnotemark[1] )~\and M.~Alighieri(\protect\footnotemark[1] )~\and B.~Tomelleri(\protect\footnotemark[1] )}
\date{2020/11/01}

\begin{document}
\maketitle

\begin{mdframed}
\textbf{Riassunto:} --- Studiamo il comportamento di diodi in silicio PN,
esplorando la propria curva caratteristica $V - I$ al di fuori dei regimi di
lavoro ordinari, tramite campionamenti digitali dei segnali ai capi del
componente. Discutiamo la presenza di una componente resistiva del diodo e ne
misuriamo l'entità, al fine di arrivare ad un modello teorico in grado
giustificare eventuali deviazioni dal modello di Shockley, verso una risposta
-ohmica- dovuta alla resistenza della giunzione PN al passaggio di correnti.\\\\
PACS 01.40.-d – Education.\\
PACS 01.50.Pa – Laboratory experiments and apparatus.
\end{mdframed}

\section{Introduzione}
L'alta resistività intrinseca al silicio, di cui è composta la giunzione
bipolare, comporta la presenza di una sua componente resistiva: questa 
risulta sempre meno trascurabile agli effetti del passaggio di corrente
attraverso il diodo, all'aumentare della tensione ai suoi capi e della sua 
temperatura. Per poter modellare la componente -ohmica- di un diodo
percorso da correnti alte si propone un modello semplice di resistenza
parassita in serie, in grado di descriverne gli effetti, verificandone 
sperimentalmente la validità e i limiti. 
\section{Metodo e apparato sperimentale}
Dovendo lavorare con correnti relativamente alte per il circuito sotto studio,
al fine di minimizzare effetti termici-dissipativi secondari ed evitare danni
all'apparato, si imprimono sui componenti correnti pulsate o di durate molto
brevi, limitando dunque il trasferimento di energia sui componenti.
\subsection{Acquisizione dati}
Si è fatto uso della piattaforma di sviluppo \verb+Teensy 3.2+\cite{teensy} per
il campionamento di segnali, essendo non solo più veloce e capiente in memoria
rispetto ad Arduino, ma oltretutto dotato di due ADC, entrambi con risoluzione
maggiore nel campionamento analogico, a 12 bit (reali). In particolare, 
\verb+Teensy+ è in grado di effettuare una lettura differenziale su due ingressi
analogici e sincronizzata tra i due ADC. Questo ci permette di misurare
pressoché -simultaneamente- la differenza di potenziale e intensità di corrente
percepite dal diodo, dunque la sua curva caratteristica. Per prima cosa
si sono misurati i rispettivi fattori di conversione $\xi_{\rm V}$ [V/digit] e
$\xi_I$ [A/digit] tramite un fit lineare sui campionamenti di tensione e
corrente continua generati da un trasformatore di d.d.p $V_0 \approx 4.95$ V.
Come illustrato sotto:
\input{./figs/thev.tex}
\subsection{Schema circuitale del sistema}
%\input{simpletikz.tex}
\begin{figure}[!htb]
	\centering 
 		\includegraphics[scale=2.2]{./simple.pdf}
 	\caption{Versione semplice del circuito \label{sch:smpl}}
\end{figure}
\begin{figure}[!htb]
	\centering 
 		\includegraphics[scale=1.3]{./gestione.pdf}
 	\caption{Circuito globale per la gestione del diodo \label{sch:gest}}
\end{figure}
\begin{figure}[!htb]
	\centering 
 		\includegraphics[scale=2.2]{./measure.pdf}
 	\caption{Schema circuitale del sistema di lettura (\texttt{Teensy})
	\label{sch:rdng}}
\end{figure}
Come ultima cautela preliminare, per minimizzare le influenze esterne sulla
misura della componente resistiva del diodo, si sono misurate le resistenze dei
collegamenti del circuito, in quanto i cavi reali e gli ingressi del diodo
possono avere resistenze non trascurabili rispetto a quella opposta dal diodo
in regime di conduzione, nell'ordine di qualche ohm.

Tramite il semplice circuito con interruttore in Figura \ref{sch:smpl} riusciamo
a ricostruire sperimentalmente la curva caratteristica V-$I$ del diodo: si
lascia scaricare repentinamente il condensatore sulla serie diodo-resistenza
chiudendo l'interruttore, dunque dai canali di un oscilloscopio si misurano le
le cadute di tensione ai capi del diodo e della resistenza $R_P$, da cui si
ricava l'intensità di corrente che scorre su entrambi i componenti tramite la
legge di Ohm.\[
I_d = \frac{\Delta V_{\mathrm{mis}}}{R_P}
.\] 

Vista la limitata attendibilità del circuito con interruttore azionato
manualmente si è costruito una seconda versione del circuito (Fig.
\ref{sch:gest}), in cui figurano: 
\begin{description}
	\item [Un sottocircuito] di switch	
	\item [Un condensatore] di capacità maggiore di un paio di ordini di 
	grandezza.
	\item [Teensy] La piattaforma di sviluppo impiegata per caricare
	(e scaricare) il condensatore a diverse tensioni e per la misura delle
	tensioni ai capi del diodo e della resistenza.
\end{description}

Si sviluppano due casi principali, dipendenti sostanzialmente dal valore della
resistenza $R_P$ posta in serie al diodo:\\

Se lasciamo caricare gradualmente il condensatore, essendo $C$ decisamente
maggiore rispetto ai condensatori precedenti si ha un tempo di carica
$\tau = RC$ abbastanza prolungato, in cui possiamo campionare contemporaneamente
tensione e corrente ai capi del diodo per correnti modeste, nel regime in cui
il diodo è "interdetto" ed oppone resistenza al flusso di carica.

Viceversa, nel regime in cui si applichi al diodo una d.d.p. ben al disopra di
$V_{\rm thr} \approx 0.6 $V, che siamo liberi di esplorare variando la tensione
di carica di C con il sottocircuito destro, il diodo idealmente lascia passare
tutta la corrente impressa sulla giunzione. Allora per caratterizzare la
risposta del diodo senza cambiarne drasticamente le caratteristiche (ad esempio
per eccessiva agitazione termica) si impiega il circuito di switch per
imprimere impulsi di alta corrente e breve durata sulla giunzione PN, di cui
misuriamo la curva caratteristica in risposta con i due ADC di \verb+Teensy+.

Se effettivamente il diodo, oltre ad una certa soglia di d.d.p. inizia ad avere
componente resistiva sempre più pronunciata, allora riportando le nostre
previsioni in carta semilogaritmica ci si aspetterebbe di trovare una retta,
entro il regime in cui è valida l'approssimazione di Shockley, ma oltre a
questo, una regione in cui la curva caratteristica della giunzione ora mostra
una dipendenza apprezzabilmente più lineare/ohmica dell'intensità di corrente
dalla $\Delta$V rispetto a prima, a cui corrisponderebbe il grafico "piatto"
di un logaritmo.  

\subsection{Cenni Teorici}
Il diodo e la resistenza $R_1$ costituiscono, nel loro insieme, un partitore
di tensione, pertanto la corrente che scorre all'interno dei due è la stessa.
Un diodo reale può essere a sua volta schematizzato come un ramo, costituito
da un resistore ohmico ed un diodo ideale in serie. Dunque ci aspettiamo che
la relazione che lega la corrente e la tensione ai capi del diodo possa essere
descritta da una legge del tipo:
\begin{equation}\label{eq: model}
	\Delta V = \Delta V_{\text{diodo}} + \Delta V_{\text{resistore}} =
	\eta V_T  \ln{\left(\frac{I+I_0}{I_0}\right)} + RI
\end{equation}

\section{Risultati e Analisi dati}
Il nostro intento è verificare che i dati raccolti siano in accordo con la
legge \eqref{eq: model}. Tramite la legge di Ohm è possibile ricavare la
corrente nel diodo, dividendo la caduta di tensione misurate dalla boccola
\verb'ADC0' per la resistenza $R$. Si è quindi effettuato un filtraggio volto
all'eliminazione degli offset e dei punti meno significativi, assumendoli quali
variabili indipendenti e di natura gaussiana. Per una discussione dettagliata
del sistema di filtraggio dati si rimanda alla sezione A delle Appendici.

Successivamente, è stato effettuato un fit sulla base dei dati selezionati.
Ingenuamente, avremo dunque potuto pensare di adottare la legge 
\eqref{eq: model} direttamente. Tuttavia ciò comporterebbe il fallimento
repentino del fit a causa dei valori negativi o nulli che debitamente si
riscontrerebbero entro il logaritmo. \`E stata quindi adottata una legge
alternativa basata sul metodo delle tangenti (o di Newton). Per una trattazione
approfondita del modello di fit, si rimanda alla sezione B delle Appendici.

I dati raccolti con sovrapposta la funzione di fit sono stati posti all'interno 
del grafico 1.
%[inserire grafico 1]
I parametri stimati dal fit risultano essere:
\begin{align*}
	R_{diodo} &= 47.5 \pm 0.2 \; \rm m\Omega \\
	\eta V_T &= 46.40 \pm 0.06 \; \rm mV \\
	I_0 &= (3.18 \pm 0.05) \; \rm nA \\
	\sigma_{R, \eta V_T} &= ? \\   
	\sigma_{R, I_0} &= ? \\
	\sigma_{I_0, \eta V_T} &= ? \\
	\chi^2/\text{ndof} &= 4483/2760 \\
	\text{abs\_sigma} &= \rm False
\end{align*}
Il $\chi^2$ risulta essere pari a 4483 contro un aspettato di 2760.
%(i dati vanno cambiati con quelli del fit fatto bene)
Successivamente, i dati e la funzione sono stati posti all'interno di un
grafico in scala semilogaritmica (grafico 2).
%[inserire grafico 2]

\section{Conclusioni}
L'andamento dei dati sperimentali non risulta essere ben descritto dalla legge
\eqref{eq: model}. Ciò può essere spiegato sulla base della non linearità dei
convertitori analogici digitali all'interno di Teensy 3.2, che ha comportato
degli errori difficilmente stimabili e presumibilmente correlati. Tuttavia, si
riscontrano delle analogie tra la funzione di fit ed i dati ottenuti. In
particolare, dal grafico in scala lineare, si osserva che a correnti alte
l'andamento risulta essere pressoch\`e lineare in accordo con l'ipotesi sulla
resistenza interna. Dal grafico in scala semilogaritmica, inoltre, osserviamo
che i dati risultano possedere un andamento simile a quello caratteristico 
della curva di Shockley per poi appiattirsi a partire da un particolare valore
della tensione, esattamente come ci aspetteremo da un andamento lineare.
Pertanto potremo aspettarci che i parametri stimati dal fit siano significativi.

\section{Appendice A: Filtraggio Dati}
Durante l’esperienza è stato raccolto un grande numero di dati, raccolti in
run diversi in base alla resistenza scelta, dunque a zone differenti della
curva. Sì è posto il problema di eliminare degli outliers in modo
indipendente dal modello scelto. Inoltre le varie serie di dati si
sovrappongono, dunque è necessario eliminare quei dati che, non aggiungendo
informazioni utili, vanno a “sporcare” il grafico.
Il sistema di filtraggio di dati implementato nell'eseguibile si compone di 2
fasi: la prima è l’eliminazione di outliers, la seconda consiste
nell'eliminazione di dati non significativi.
\subsection{Introduzione}
%In generale possiamo immaginare che ad ogni misura sia associata una forma
%quadratica rappresentata dalla matrice di covarianza della stessa, cioè che
%durante la misura si commetta un errore statistico normale noto. Possiamo poi
%immaginare che il misurando abbia un’altra matrice di covarianza. Nel
%procedimento proposto questo potrebbe essere teoricamente trattato per esteso,
%tuttavia non si troverebbe una forma chiusa generale per il problema in
%questione.
Supponiamo che gli errori su $x$ e $y$ siano indipendenti
e che $\sigma_x ^2 \coloneqq \var{x}$ sia nota a priori e che abbia 
distribuzione gaussiana per ogni misura. Questo non è in generale vero, ma 
questa ipotesi può essere trascurata quando la correlazione tra le varianze
delle misure su $x$ e $y$ sono indipendenti dai valori assunti dalle $x$ e $y$
stesse e vi sono “molti” dati entro una deviazione lungo $x$. In questo caso,
infatti, la correlazione viene inclusa nella varianza lungo $y$ e, dal
teorema del limite centrale si vede che la non-normalità delle distribuzioni
è trascurabile.
In ogni modo, nei nostri dati x e y risultano ragionevolmente indipendenti,
quindi l'assunzione dovrebbe essere lecita.
\subsection{Procedimento}
Supponiamo, per ogni punto, che la distribuzione sia gaussiana secondo una
matrice di covarianza diagonale nella base $\{x, y\}$ : allora la densità di
probabilità che un punto che abbia misurato $x$ si trovi a tale ascissa $x_i$
si ricava banalmente integrando lungo $y$ a $x$ fissata:
\[
	\ud P = \frac{1}{\sigma_{x_i} \sqrt{2\pi}}
	e^{-\frac{1}{2}{\frac{(x - x_i)^2}{\sigma_{x_i}^2}}} \ud x
.\] 
Dunque, ripetendo più volte la stessa misura, la probabilità
\[
	P\left( \mid x - x_i \mid  \leq \eps  \right) = \eps G_{x_i} 
.\]
dove \[
	G_{x_i} \coloneqq \frac{1}{\sigma_{x_i} \sqrt{2\pi}}
	e^{-\frac{1}{2}{\frac{(x - x_i)^2}{\sigma_{x_i}^2}}}
.\] 
e $\eps > 0$ è piccolo a piacere. Scegliendo allora solo quelle misure x per
cui vale $\mid x - x_i \mid \leq \eps$, queste saranno in numero intorno a:
\[
	N_i \coloneqq N_{\text{tot}} \frac{G_{x_i}}{\sum_j G_{x_j}} =
		N_{\text{tot}} w_i
.\] 
che definisce implicitamente i pesi $w_i$ con cui si mediano le distribuzioni
di probabilità gaussiane $G_{x_i}$.
Allora, detto:
\[
	G_{y_i} \coloneqq \frac{1}{\sigma_{y_i} \sqrt{2\pi}}
	e^{-\frac{1}{2}{\frac{(\mu_y - y_i)^2}{\sigma_{y_i}^2}}}
.\] 
Per il principio di massima verosimiglianza siamo quindi interessati a
massimizzare la quantità:
\[
	\like = \prod_{i=1}^{n} \prod_{j=1}^{N_i} G_{y_i} = 
	\prod_{i=1}^{n} G_{y_i}^{N_i}
.\] 
Per la monotonia del logaritmo il problema equivale a massimizzare la quantità:
\[
	\ln{\like} = \sum_{i=1}^{n}\ln{G_{y_i}}^{N_{\text{tot}}w_i} = 
	\frac{N_{\text{tot}}} {\sum_{j=1}^{n} G_{x_j}} 
	\sum_{i=1}^{n} G_{x_i} \ln{G_{y_i}}
.\] 
Per cui, a meno di costanti risulta:
\begin{equation}\label{eq: likeconst}
	\ln{\like} - \text{const.} \propto \sum_{i=1}^{n} -G_{x_i} \ln{\sigma_y}
	- \frac{1}{2} G_{x_i} \left( \frac{y_i - \mu_y}{\sigma_y} \right)^2
\end{equation}
Imponendo la condizione di stazionarietà rispetto a $\mu_y$ si ottiene dunque:
\begin{equation}\label{eq: muy}
	\mu_y = \sum_{i=1}^{n} y_i w_i 
\end{equation} 
Una volta sostituito in \eqref{eq: likeconst} quanto appena trovato per $\mu_y$
e imponendo la stessa condizione di stazionarietà rispetto a $\sigma_y$ si ha:
\begin{equation}\label{eq: sigmay}
	\sigma_y^2 = \sum_{i=1}^{n} (y_i - \mu_y)^2 w_i .
\end{equation}
Infine è possibile ricavare la varianza di $\mu_y$ dalla definizione di valore
di aspettazione, riconducendola più volte a integrali di gaussiane di altezze
e ampiezze diverse:
\begin{align}
	\var{\mu_y} &= \sum_{i=1}^{n} \left[w_i^2 \sigma_y^2 + 
	\left(\frac{y_i}{\sum_{j=1}^{n} w_j}  \right)^2 \frac{
	\frac{e^{-\frac{(x-x_i)^2}{3 \sigma_x^2}}} {\sigma_x \sqrt{6 \pi} } +
	\frac{e^{-3\frac{(x-x_i)^2}{4 \sigma_x^2}}} {\sigma_x \sqrt{\pi} } +
	\frac{e^{-\frac{(x-x_i)^2}{\sigma_x^2}}} {\sigma_x \sqrt{2 \pi}}
	} {\sqrt{2 \pi}} \right] =\\ \label{aln: varmuy}
        &= \sum_{i=1}^{n} \left[w_i^2 \sigma_y^2 + 
	\left(\frac{y_i}{\sum_{j=1}^{n} w_j}  \right)^2 \frac{
	e^{-\frac{(x-x_i)^2}{3 \sigma_x^2}} +  
	\sqrt{3}\left( e^{-\frac{(x-x_i)^2}{\sigma_x^2}} -
	\sqrt{2} e^{-3 \frac{(x-x_i)^2}{4 \sigma_x^2}} \right)
	} {2 \sqrt{3}\pi \sigma_x^2} \right]
\end{align}
Riassumendo:\\
Nella \eqref{eq: muy} prendiamo una media dei campionamenti intorno ad un'
ascissa $x$ in esame, pesata sulla distanza che gli $x_i$ hanno da questa. 
Effettivamente quello che stiamo facendo è un stima di densità di kernel,
per cui consideriamo i punti come -sfocati- da un \emph{blur gaussiano};
dove però nel nostro caso riscaliamo la stima in base al valore assunto da $y$.
Lo stesso ragionamento vale per $\sigma_y^2$, si ha una stima della varianza
dei dati la variare di y, pesata sulla distanza dai valori studiati. Dunque
$\mu_y \pm \sigma_y$ ci dà una stima della distribuzione dei nostri dati.
\subsection{$\var{\mu_y}$}
Mentre $\sigma_y$ rappresenta la distribuzione dei dati intorno al valor medio 
$\mu_y$, $\var{\mu_y}$ ci dà un'idea dell’incertezza che attribuiamo alla
miglior stima di y. Questo ci è utile per determinare la convergenza della
stima in funzione dei dati acquisiti.
Infatti: più la densità dei dati è grande rispetto alla deviazione standard
$\sigma_x$, più la stima del valore centrale risulta precisa. Graficamente
la banda di confidenza è più ristretta dove si concentrano i dati, viceversa
tende ad allargarsi dove i dati sono sparsi, a distanze paragonabili a
$\sigma_x$. Dalla seconda somma nell'espressione \eqref{aln: varmuy} segue
allora che la stima del valore centrale è statisticamente significativa solo
quando si media su un intervallo campionato con almeno qualche punto ogni
deviazione $\sigma_x$, altrimenti $\var{\mu_y}$ tende a $+\infty$ come 
$\sim e^{x^2}$in assenza di dati, dove non è possibile stabilire con
precisione il valore di $\mu_y$.
\subsection{Filtro outliers}
La parte più semplice nel filtraggio dati consiste nello scartare tutti quei
punti che distano da $\mu_y$ più di una soglia arbitraria $k$ di deviazioni
standard dalla media $\sigma_y$, ovvero quegli $y_i$ per cui,
indipendentemente dal modello di fit vale $|y_i - \mu_y| \geq k\sigma_y$.
\subsection{Filtro dati non significativi}
Supponiamo di avere 2 set di dati fatti con diverse resistenze, il primo con
una resistenza bassa, il secondo con una alta: Il primo set esplorerà
la regione ad alta corrente, mentre il secondo la regione di basse correnti.
Però il primo insieme conterrà, in generale, anche campionamenti delle zone
basse, ma su queste fornirà dei valori meno significativi: Esponiamo dunque
il criterio sviluppato per ridurre l'influenza di questi punti meno
significativi sulla ricerca dei parametri di best-fit e sulla rappresentazione 
finale dei dati.\\
Per capire se in un certo punto i dati di $A$ sono significativi, calcoliamo
la misura di significatività che abbiamo sviluppato in \eqref{aln: varmuy}$:
\var{\mu_y}$ di $A$ e di $B$. Perciò se $\var{\mu_y}$ di $A$ è maggiore di 
$q\var{\mu_y}$ di $B$, con $q$ arbitrario, questo indica che i dati di $A$
ci stanno dando meno informazioni rispetto a quelli di $B$. A questo punto
è sufficiente controllare tutti i punti scorrendo su tutte le combinazioni
di set per eliminare i dati non significativi, che rendono meno
immediata l'interpretazione il grafico.

\section{Appendice B: Metodo di Fit}
Partiamo dall'osservazione che la tensione ai capi del diodo può essere
scomposta in $\Delta V_r$ e $\Delta V_d$ come illustrato in figura:

dove il resistore a sinistra indica proprio la resistenza di un diodo reale,
mentre il diodo alla sua destra rappresenta una giunzione bipolare PN, ossia
il diodo ideale descritto dall’equazione di Shockley.
Dunque ci aspettiamo che la tensione $V$ campionata con \verb+Teensy+ ai capi
del diodo rispetti la legge:
\begin{equation}\label{eq: invsck}
V = \Delta V = \Delta V_d + \Delta V_r = \eta V_T = \ln{\frac{I+I_0}{I_0}} + RI
\end{equation}
All’interno dello script \verb'fit_file.py', voltages rappresenta il vettore
delle ascisse dei punti campionati, mentre currents rappresenta l'analogo per
le ordinate. Le incertezze associate a questi due all'interno dello script
prendono i nomi \verb'voltageErrs' e \verb'currentErrs'.
Ingenuamente potremmo essere tentati di effettuare una sorta di fit “inverso”
(delle x in funzione delle y).
Tuttavia una tale operazione non è affatto banale, a causa dei valori 
negativi/nulli che debitamente si riscontreranno all’interno del logaritmo
nella legge \eqref{eq: invsck}, che possono facilmente portare al fallimento
dell'algoritmo di fit.
\`E per questo che si è dovuto definire il modello "inverso" nella forma di
una legge di natura approssimata per ricorsione, grazie al metodo di Newton. 
Al livello di implementazione si è quindi dichiarata una funzione:
\begin{lstlisting}[label={lst: curr}]
def curr(V, I0, nVt, R):
    v = V
    for i in range(Nstep):
	a = deriv_errFun(v, I0, nVt, R)
	v = v - errFun(v, V, I0, nVt, R) /a
    return (V - v)/R
\end{lstlisting}

Questa verrà in seguito usata come modello per la corrente in funzione della
tensione ai capi del diodo reale nella routine di fit per minimi quadrati.
La funzione \verb'curr' cerca di -linearizzare- la relazione tra $I$ e $V$
nell'intorno di un determinato $V$. Come sappiamo infatti dal metodo di Newton:
detta $f(x)$ una funzione tale che $f(x) = 0$ e, dato un valore iniziale
tale che $f(x[0]) = \alpha$ generico, sappiamo che la relazione di ricorsione
\begin{align}
	x[0] = \alpha \\ \label{aln: newton}
	x[N+1] = x[N] - \frac{f(x[N])}{f'(x[N])}
\end{align}
converge ad un valore approssimato per la nostra $x$, commettendo un errore 
sempre più piccolo al crescere di $N$, indice del livello di ricorsione
raggiunto. 
Nel nostro caso $x$ sarà quella tensione $v$ per cui vale l'identità:
\begin{equation}\label{eq: sck}
	I(V) \coloneqq I_0^{\frac{\Delta V}{\eta V_T}} = \frac{V - v}{R}	
\end{equation}
che identifica il tratto lineare descritto da una retta con coefficiente
angolare negativo pari a $-\frac{1}{R}$ ed intercetta $I = v/R$ 
sull’asse delle ordinate; si tratta proprio della retta di carico del diodo.
L'equazione ricorsiva \eqref{aln: newton} ci permette quindi di determinare
il punto di lavoro $v$ per cui la corrente (di lavoro) si può scrivere come
$\frac{V - v}{R}$, come si vede dal return nella funzione di fit
\verb'curr(V, parametri liberi)'.
A questo punto si può già intuire quali siano le altre due funzioni di
appoggio richiamate dentro \verb'curr', di cui riportiamo le definizioni
per completezza:
\begin{lstlisting}[label={lst: errFun}]
def sck(V, I0, nVt):
    return I0*(pylab.exp(V/nVt) - 1)

def errFun(V, V0, I0, nVt, R):
    return sck(V, I0, nVt) + (V - V0)/R

def deriv_errFun(V, I0, nVt, R):
    return I0 / nVt * pylab.exp(V/nVt) + 1./R
\end{lstlisting}
Infatti \verb'errFun' e \verb'deriv_errFun' sono, rispettivamente, la relazione 
da minimizzare specificata dalla legge \eqref{eq: sck} e la sua derivata
rispetto a $V$.
In conclusione, si è effettuato un fit dei minimi quadrati implementato in
\verb+Python+ mediante la funzione \emph{curve\_fit} dal modulo \verb+optimize+
interno alla libreria \texttt{Scipy}\cite{scipy} adottando come modello
\verb'curr' e lasciando liberi tutti i suoi parametri.
In particolare si è tenuto conto dell'incertezza sulla variabile indipendente
con il metodo "dell'errore efficace": ovvero propagando gli errori 
\verb'voltageErrs' sulle $V$, tramite le stime dei parametri ottenute da un fit
preliminare, in cui prendiamo \verb'currentErrs' come sole incertezze sulla
variabile dipendente $I$. Lo stesso algoritmo di fit viene iterato più volte,
però assumendo come incertezza sulle misure una sorta di errore efficace 
dato dalla somma in quadratura dei due contributi all'incertezza: 
\begin{equation}\label{eq: seff}
	\Delta_{\text{eff}_i} \coloneqq \sqrt{\left|\frac{\ud \, curr}{\ud V}
	\right|_{V = V_i} \Delta_{V_i}^2 + \Delta_{I_i}^2}.
\end{equation}
fin a che non converge ai valori ottimali dei parametri.
\medskip
\bibliographystyle{IEEEtrandoi}
\bibliography{refs}
\end{document}
