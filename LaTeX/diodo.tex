\documentclass{article}[a4paper, oneside ,11pt]
\usepackage[T1]{fontenc}
\usepackage[utf8]{inputenc}
\usepackage{calc}
\usepackage{amsmath,amssymb,amsthm, thmtools, amsfonts}
\usepackage[nochapters,pdfspacing]{classicthesis}
%\usepackage{hyperref}% clashes with classicthesis
\usepackage{cleveref}
\usepackage[siunitx]{circuitikz}
\usepackage{booktabs}
\usepackage{graphicx}
\usepackage{caption}
\usepackage{geometry}
\usepackage{float}
\usepackage{mdframed}
\usepackage{xcolor}
\usepackage{siunitx}
\usepackage[italian]{babel}
\usepackage{pgfplots}
\usepackage{titling}
\usepackage{listings}
\usepackage{lmodern}
\usepackage{url}
\usepgfplotslibrary{external} 
\tikzexternalize

\pgfplotsset{compat=1.16}
\lstset{
basicstyle=\ttfamily,
columns=fullflexible,
keepspaces=true,
}
\mdfsetup{linewidth=0.6pt}
\graphicspath{{./figs/}}
\makeatletter
\def\input@path{{./figs/}}
%or: \def\input@path{{/path/to/folder/}{/path/to/other/folder/}}
\makeatother

\input{./math}

\geometry{a4paper, left=25mm, right=25mm, top=30mm, bottom=30mm}
\title{Modellizzazione della resistenza di diodi a giunzione PN per alte correnti di lavoro}
\author{L. Ciucci(\thanks{Dipartimento di Fisica E. Fermi, Universit\`a di Pisa - Pisa, Italy} ) \and S. Bruzzesi(\protect\footnotemark[1] ) \and B. Tomelleri(\protect\footnotemark[1] )}
\date{2020/11/01}

\begin{document}
\maketitle

\begin{mdframed}
\textbf{Riassunto:} --- Studiamo il comportamento di diodi in silicio PN, esplorando la propria curva caratteristica $V - I$ al di fuori dei regimi di lavoro ordinari, tramite campionamenti digitali dei segnali ai capi del componente. Discutiamo la presenza di una componente resistiva del diodo e ne misuriamo l'entità, al fine di arrivare ad un modello teorico in grado giustificare eventuali deviazioni dal modello di Shockley, verso una risposta -ohmica- dovuta alla resistenza della giunzione PN al passaggio di correnti.\\\\
PACS 01.40.-d – Education.\\
PACS 01.50.Pa – Laboratory experiments and apparatus.
\end{mdframed}

\section{Introduzione}
Per poter modellare il comportamento di un diodo percorso da correnti alte, si propone un modello semplice in grado di spiegarne il comportamento -ohmico-, verificandone sperimentalmente la validità e i limiti.
\section{Metodo e apparato sperimentale}
Dovendo lavorare con correnti relativamente alte per il circuito sotto studio, al fine di minimizzare effetti termici-dissipativi secondari ed evitare danni all'apparato, si imprimono sui componenti correnti pulsate o di durate molto brevi, limitando dunque il trasferimento di energia.
\subsection{Acquisizione dati}
Perciò si è fatto uso della piattaforma di sviluppo \verb+Teensy 3.2+\cite{teensy} per il campionamento di segnali, essendo non solo più veloce e capiente in memoria rispetto ad Arduino, ma oltretutto dotato di due ADC, entrambi con risoluzione maggiore nel campionamento analogico, a 16 bit. Questo ad esempio ci permette di misurare -simultaneamente- la differenza di potenziale e intensità di corrente, dunque la curva caratteristica del diodo in esame. 
\subsection{Schema circuitale del sistema}
\begin{center}
\begin{circuitikz}[american]
\draw (0,0)
	to[voltmeter, l=$\Delta V_{\rm mis}$] (0,4)
	to[short] (2,4)
	to[R=$R_{\ud}$, *-] (2,2)
	to[Do, -*] (2,0)
	to[short] (0,0);
	\draw (2,0)
	to[short] (5,0)
	to[V] (5,6) % The current source
	to[spst] (2,6)
	to[C=100<\farad>] (2,4);
\end{circuitikz}
\end{center}
Infine, volendo misurare la componente resistiva del solo diodo è fondamentale conoscere la resistenza dei collegamenti del circuito ed, eventualmente stabilire sotto quali condizioni essi risultino trascurabili.
\section{Risultati e Analisi dati}
\subsection{Nota sul metodo di fit}
Per determinare i parametri ottimali e le rispettive varianze si \`e implementato un metodo di fit basato sui minimi quadrati in \verb+Python+ mediante la funzione \emph{curve\_fit} della libreria Scipy\cite{scipy}.
\medskip
\bibliographystyle{IEEEtrandoi}
\bibliography{refs}
\end{document}
