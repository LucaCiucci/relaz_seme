\documentclass{article}[a4paper, oneside, 11pt]
\usepackage[T1]{fontenc}
\usepackage[utf8]{inputenc}
\usepackage{calc}
\usepackage{amsmath, amssymb, amsthm, thmtools, amsfonts}
\usepackage{mathtools}
\usepackage[nochapters,pdfspacing]{classicthesis}
%\usepackage{hyperref}% clashes with classicthesis
\usepackage{cleveref}
\usepackage[siunitx]{circuitikz}
\usepackage{booktabs}
\usepackage{graphicx}
\usepackage{caption}
\usepackage{subcaption}
\usepackage{geometry}
\usepackage{float}
\usepackage{mdframed}
\usepackage{xcolor}
\usepackage{siunitx}
\usepackage[italian]{babel}
\usepackage{pgfplots}
\usepackage{titling}
\usepackage{listings}
\usepackage{lmodern}
\usepackage{url}
\usepgfplotslibrary{external} 
\tikzexternalize

\pgfplotsset{compat=1.15}
\lstset{
language=Python,
basicstyle=\ttfamily,
columns=fullflexible,
keepspaces=true,
}
\mdfsetup{linewidth=0.6pt}
\graphicspath{{./figs/}}
\makeatletter
\def\input@path{{./figs/}}
%or: \def\input@path{{/path/to/folder/}{/path/to/other/folder/}}
\makeatother

\input{./math}

\geometry{a4paper, left=25mm, right=25mm, top=25mm, bottom=25mm}
\title{Modellizzazione della resistenza di diodi a giunzione PN per alte 
correnti di lavoro}
\author{L.~Ciucci(\thanks{Dipartimento di Fisica E.~Fermi, Universit\`a di Pisa 
- Pisa, Italy} )~\and S.~Bruzzesi(\protect\footnotemark[1] )~\and 
M.~Romagnoli(\protect\footnotemark[1] )~\and 
M.~Alighieri(\protect\footnotemark[1] )~\and 
B.~Tomelleri(\protect\footnotemark[1] )}
%\date{2020/11/01}

\begin{document}
\maketitle

%================================================================
%                            Riassunto
%================================================================
%\begin{mdframed}
%\textbf{Riassunto:} --- Studiamo il comportamento di diodi in silicio PN,
%esplorando la propria curva caratteristica $V - I$ al di fuori dei regimi di
%lavoro ordinari, tramite campionamenti digitali dei segnali ai capi del
%componente. Discutiamo la presenza di una componente resistiva del diodo e ne
%misuriamo l'entità, al fine di arrivare ad un modello teorico in grado
%giustificare eventuali deviazioni dal modello di Shockley, verso una risposta
%-ohmica- dovuta alla resistenza della giunzione PN al passaggio di 
%correnti.\\\\
%PACS 01.40.-d – Education.\\
%PACS 01.50.Pa – Laboratory experiments and apparatus.
%\end{mdframed}


%================================================================
%                         Introduzione
%================================================================
\section{Introduzione}
L'alta resistività intrinseca al silicio, di cui è composta la giunzione
bipolare, comporta la presenza di una sua componente resistiva: questa 
risulta sempre meno trascurabile agli effetti del passaggio di corrente
attraverso il diodo, all'aumentare della tensione ai suoi capi.
Per poter modellare la componente -ohmica- di un diodo
percorso da correnti alte si propone un modello semplice di resistenza
parassita in serie, in grado di descriverne gli effetti, verificandone 
sperimentalmente la validità.


%================================================================
%                         Cenni teorici
%================================================================
\section{Cenni Teorici}
Secondo le ipotesi, un diodo reale può essere schematizzato quale un resistore 
ohmico ed un diodo ideale in serie. Pertanto ci aspettiamo che la corrente che 
attraversa tali elementi sia la medesima:
\begin{equation}
	I = \frac{\Delta V_{\text{resistore}}}{R} =
	I_0 \left( e^{\frac{\Delta V_{\text{diodo}}}{\eta V_t}} - 1\right)
\end{equation}
Dunque la relazione che lega la corrente e la tensione ai capi del diodo può 
essere espressa secondo la legge:
\begin{equation}\label{eq: model}
	\Delta V = \Delta V_{\text{diodo}} + \Delta V_{\text{resistore}} = \eta 
V_T \ln{\left(\frac{
	I+I_0}{I_0}\right)} + RI
\end{equation}
\begin{figure}[!htbp]
	\centering 
 		\includegraphics[scale=0.25]{./confronto_curve.png}
 	\caption{confronto degli andamenti: la curva di Shockley (in rosso) e la 
legge \eqref{eq: model} (in viola)}
\end{figure}


%================================================================
%                Metodo e apparato sperimentale
%================================================================
\section{Metodo e apparato sperimentale}
Il circuito sotto studio lavora con correnti relativamente alte. Onde evitare 
sostanziali aumenti di temperatura nei componenti ed eventuali danni agli 
stessi si imprimono correnti impulsate.
La durata degli impulsi è inferiore ai 200 $\mu$s, che 
corrisponde ad un'energia impressa inferiore a 1 mJ (dunque ad un aumento della 
temperatura del semiconduttore inferiore a 0.5 K).
%Dovendo lavorare con correnti relativamente alte per il circuito sotto studio,
%al fine di minimizzare effetti termici-dissipativi secondari ed evitare danni
%all'apparato, si imprimono sui componenti correnti pulsate o di durate molto
%brevi, limitando dunque il trasferimento di energia sui componenti.

%================================
%           Apparato
%================================
\subsection{Apparato}
L'apparato sperimentale è costituito da un circuito, realizzato su basetta 
sperimentale, il cui scopo è generare correnti impulsate attraverso il diodo 
$D1$ e la resistenza $R1$. La misura della differenza di potenziale ai capi di 
quest'ultima
permette la stima della corrente che attraversa il diodo. Ai fini di esplorare 
ampie zone 
di lavoro del diodo, la resistenza è stata cambiata opportunamente scegliendo 
tra le seguenti:
\begin{table}[H]% la tabella deve stare per forza sotto "seguenti:" oopure bisgna camiare la frase
    \begin{center}
	\begin{tabular}{lll}
	    \toprule
	    $R1$ nom. [$\Omega$] & $R1$ mult. [$\Omega$] \\ 
	    \midrule
	    \midrule
	    $0.22 \pm 3 \% $         & $0.226 \pm 0.008$ \\
	    $2.2 \pm 5 \% $          & $2.212 \pm 0.008$ \\
	    $22 \pm 5 \% $           & $21.86 \pm 0.010$ \\ 
	    $220 \pm 5 \% $          & $216.22 \pm 0.07$ \\
	    $2.2 k \pm 5 \% $        & $2202.1 \pm 0.4$ \\
	    $22 k \pm 5 \% $         & ($21.7 \pm 0.3)10^3$ \\
	    $0.22 M \pm 5 \% $        & ($217 \pm 3)10^3$ \\
	    \bottomrule
	\end{tabular}
	\caption{I valori delle resistenze poste in serie al diodo, riportate in
		valore nominale e misurate con multimetri digitali. \label{tab:res}}
    \end{center}
\end{table}
Nel circuito può essere individuata una parte che si occupa della carica e 
scarica controllata del condensatore $C1$, il cui scopo è quello di fornire 
una tensione che possa essere facilmente regolata e sufficientemente stabile 
durante un impulso. Un'altra parte del circuito si occupa di collegare a 
comando il condensatore alla serie $R1-D1$ attraverso un MOS-FET.\\
Il circuito è alimentato da 2 tensioni fornite da un alimentatore stabilizzato 
switching e da un Buck Boost Converter. Un segnale fornito su $T2$ provoca la 
carica del condensatore mentre su $T3$ la sua scarica. Un segnale (invertito) 
su $T1$ innesca l'impulso di corrente sul diodo.\\
La tensione di C$1$ è misurata attraverso un partitore di tensione collegato 
ad \verb+OUT3+ ed alla scheda di controllo, mentre la tensione ai punti
\verb+OUT1+ e \verb+OUT2+ è letta direttamente.\\
Come MCU per la gestione dell'apparato è stata scelta la scheda 
\verb+Teensy 3.2+\cite{teensy}. Questa si occupa del controllo dei segnali e 
delle letture analogiche. In particolare \verb+Teensy+ permette la lettura
analogica sincronizzata differenziale veloce, essendo dotato di due ADC,
entrambi con una risoluzione reale di 12 bit. La lettura differenziale è
fondamentale per acquisire coppie di dati per le tensioni ai capi del diodo
e della resistenza.

\begin{figure}[H]% so che è brutto che lascia un po' di spazio bianco ma altrimenti lo metterebbe dentro calibrazione
	\centering 
 		\includegraphics[scale=0.5]{./gestione.png}
 	\caption{Circuito globale per la gestione del diodo \label{sch:gest}}
\end{figure}

\begin{figure}[H]
	\centering 
 		\includegraphics[scale=0.5]{./measure.png}
 	\caption{Schema circuitale del sistema di lettura (verso 
	\texttt{Teensy})
	\label{sch:rdng}}
\end{figure}

%================================
%         Calibrazione
%================================
\subsection{Calibrazione}\label{sec: cal}

La calibrazione dei canali \verb+ADC0+ e \verb+ADC1+ è stata effettuata
sulla base della lettura differenziale rispettivamente fra i pin \verb+A10+
e \verb+A11+ e fra \verb+A12+ e \verb+A13+. E’ stata calibrata la lettura
di ciascuna delle due coppie separatamente.

I pin volti alla lettura della tensione maggiore sono stati collegati al 
centrale di un potenziometro, a sua volta collegato ad un generatore di 
potenziale con tensione in ingresso pari a 3.3V . Gli altri due pin sono stati 
invece collegati a massa.

Variando la resistenza e conseguentemente la differenza di potenziale in 
ingresso, è stato possibile calcolare la media, la deviazione standard 
campione e la deviazione standard dalla media relative a ciascuna tensione per 
un gruppo di 10000 misure effettuate con \verb+Teensy+ (vedi 
\href{https://github.com/LucaCiucci/relaz_seme/blob/master/sketches/teensy_calib/teensy_calib.ino}{relativo programma}).
Il centrale del potenziometro e la 
massa sono state, inoltre, collegate ad un multimetro digitale, così da avere 
il rispettivo valore in Volt per ciascuna tensione valutata in unità 
arbitrarie. Sono stati, quindi, associati degli errori a queste ultime misure  
tenendo presente le specifiche del multimetro.

Dopodiché le misure relative a ciascuna coppia di boccole sono state poste 
all’interno di un grafico con i dati raccolti da \verb+Teensy+ in ordinate e i 
valori misurati col multimetro in ascisse. Dunque è stato effettuato un fit 
lineare attraverso la legge:
\begin{equation}
y(x; m, q)  = m x + q
\end{equation}

ove $m$ e $q$ sono i parametri liberi. Come errori associati alle letture
digitali sono state utilizzate le deviazioni standard dalla media. I fit
relativi ad \verb+ADC0+ ed \verb+ADC1+ hanno conseguito i seguenti risultati:
\begin{align*}
	&\qquad \texttt{ADC0}	&&\qquad  \texttt{ADC1} \\
	m_0 &= 0.7968 \pm  0.0033  \; \mathrm{mV/digit} 
	& m_1  &= 0.7954 \pm 0.0026  \; \mathrm{mV/digit} \\
	q_0 &= - 0.18 \pm 0.34  \; \mathrm{mV} 	
	&q_1 &= 3.84  \pm 0.26  \; \mathrm{mV} \\
	\var{m_0}  &= 1.1 \cdot 10^{-11}  \; \mathrm{V}^2 /\mathrm{digit}^2
	&\var{m_1} &= 7.0 \cdot 10^{-12}  \; \mathrm{V}^2 /\mathrm{digit}^2 \\
	\var{q_0} &=1.2 \cdot 10^{-7} \; \mathrm{V}^2
	&\var{q_1} &= 6.8 \cdot 10^{-8} \; \mathrm{V}^2 \\
	\cov{m_0} {q_0} &= - 1.6 \cdot 10^{-10} \; \mathrm{V}^2 /\mathrm{digit}
        &\cov{m_1}{q_1} &= - 6.1 \cdot 10^{-11} \;
	\mathrm{V}^2 /\mathrm{digit} \\
	\chi^2/\text{ndof} &= 145/15	&\chi^2/\text{ndof} &= 96/15 \\ 
	\text{abs\_sigma} &= \rm False	&\text{abs\_sigma} &= \rm False
\end{align*}

\begin{figure}[H]% qui potremo fare qualcosa per assicurarci che le questi valori e le immagini vengano sempre tenute insieme, ma non lo so fare e non  importante
\centering
\begin{subfigure}{.5\textwidth}
	\centering 
 		\includegraphics[scale=0.5]{./digitvsvolt_ADC0_1.png}
	\label{fig: nofilter}
\end{subfigure}%
\begin{subfigure} {.5\textwidth}
	\centering 
		\includegraphics[scale=0.5]{./digitvsvolt_ADC1_1.png}
	\label{fig: filtered}
\end{subfigure}
\end{figure}

I coefficienti così riscontrati ci permetteranno di fornire una buona stima 
dei valori centrali relativi alla misura in volt delle letture analogiche 
(digit). Il $\chi^2$ risulta essere sovrastimato in quanto il MCU non ci 
fornisce una risposta perfettamente lineare ai segnali in ingresso. Dalle 
specifiche di \verb+Teensy 3.2+, sappiamo che lo scostamento dall’andamento lineare 
può essere quantificato col $7 \%$ della lettura in digit.\newline
Per maggiori informazioni riguardo alle stime delle incertezze nelle 
conversioni si rimanda direttamente al 
\href{https://github.com/LucaCiucci/relaz_seme/blob/master/Cartella_fit/funzioni.py}
{relativo script}.

%================================
%       Acquisizione dati
%================================
\subsection{Acquisizione dati}
Per ciascun valore scelto della resistenza $R1$ è stata eseguita una presa 
dati automatizzata secondo una 
\href{https://github.com/LucaCiucci/relaz_seme/blob/master/sketches/teensy_differenziale_definitivo/teensy_differenziale_definitivo.ino}{routine} programmata 
in \verb+Teensy+: il condensatore viene caricato ad una tensione prefissata;
una volta raggiunto il valore prestabilito viene inviato il segnale su $T1$,
si avvia l'acquisizione sincronizzata e si attende un tempo 50 $\mu$s per
permettere al MOS-FET di entrare in conduzione ($R2$ è stata scelta di 1k,
dunque c'è un apprezzabile ritardo tra segnale e impulso) e scartare eventuali
segnali spuri; a questo punto si inizia a memorizzare una serie da 100 coppie
di letture -sincronizzate- (a meno di sfasamenti nell'ordine delle centinaia di 
nanosecondi); si ferma l'impulso e si trasmettono i dati al computer; se la 
tensione al punto \verb+OUT2+ ha raggiunto valori maggiori di 3.3V 
l'acquisizione si ferma per non danneggiare \verb+Teensy+, altrimenti si
ricomincia caricando il condensatore ad una tensione più alta.

%================================================================
%                   Analisi dati e Risultati
%================================================================
\section{Analisi dati e Risultati}
L’intento della presente analisi è quello di verificare che i dati raccolti 
siano in accordo con la legge \eqref{eq: model}. A tale riguardo, come 
operazione preliminare è stato necessario convertire le letture analogiche 
nelle opportune coppie corrente-tensione relative al diodo. Dunque i dati sono 
stati convertiti in tensione secondo le calibrazioni dei relativi canali (
\verb+ADC0+ e \verb+ADC1+). Tale operazione è stata effettuata considerando
opportunamente gli errori come descritto nel paragrafo \ref{sec: cal}.
Successivamente, le tensioni relative ai capi della resistenza $R$ hanno
permesso di stimare la rispettiva corrente di lavoro attraverso la legge di Ohm. 
Si è quindi effettuato un filtraggio volto all'eliminazione degli outliers e 
dei punti meno significativi, assumendoli quali variabili indipendenti e di 
natura gaussiana. Per una discussione dettagliata si rimanda all'\nameref{app: 
A}. E’ necessario sottolineare che, all’interno della stessa appendice, 
$\sigma_x$ rappresenta la varianza delle letture e non gli errori associati ad 
esse.
Successivamente, è stato effettuato un fit ai minimi degli scarti quadratici. 
Dunque avremo potuto pensare di adottare la legge \eqref{eq: model} 
direttamente. Tuttavia ciò comporterebbe il fallimento del fit a causa dei 
valori negativi o nulli che si riscontrerebbero nell’argomento del logaritmo. 
\`E stata quindi utilizzata la funzione inversa come modello di fit, la quale 
è stata ricavata numericamente secondo il metodo delle tangenti (o di Newton). 
Per una trattazione approfondita, si rimanda all'\nameref{app: B}.

Da un controllo visivo dei dati si osserva che nei punti raccolti con la resistenza da 220k è presente del rumore molto maggiore delle inceretzze sulle misure. Queso è probabilmente dovuto all'influenza del circuito di lettura stesso sulle piccole correnti in gioco. Non potendo dunque attribuire degli errori significativi per questa serie, tali dati non sono stati utilizzati nella procedura di fit eseguita successivamente in quanto avrebbero avuto un peso sovrastimato in confronto agli altri dati, influenzando così i risultati finali.
 
I dati raccolti con sovrapposta la funzione di fit sono stati posti all'interno 
del grafico \ref{fig: sck_lin}.
I parametri stimati dal fit risultano essere:

\begin{align*}
	R_{\text{diodo}} &= 46.1118 \pm 0.0072 \; \rm m\Omega \\
	\eta V_T &= 47.5786 \pm 0.0033 \; \rm mV \\
	I_0 &= 4.5180 \pm 0.0043 \; \rm nA \\
	\text{offset} &=\; - 2.2043\pm 0.0073 \;  \mu A\\
	\chi^2/\text{ndof} &= 72207/251066 \\
	\text{abs\_sigma} &= \rm False
\end{align*}

\begin{figure}[H]% questi due grafici sono più problematici, potrebbe tornare utile usare vspace negativo e ritagliare le figure manualmente se necessario
	\centering 
	\includegraphics[scale=0.7]{diode_linear.png}%{./Figure_4_2_Nskip_100_NO_220k.png}
	\caption{Dati acquisiti e funzione di best fit \eqref{eq: model}. E' 
	stato rappresentato un punto ogni 100 per comodità di visualizzazione.
	\label{fig: sck_lin}}
\end{figure}

\begin{figure}[H]
	\centering 
	\includegraphics[scale=0.7]{diode_semilog.png}%{./Figure_3_1_Nskip_10.png}
	\caption{Dati acquisiti e funzione di best fit \eqref{eq: model} in 
	scala semilogaritmica. A scopo illustrativo sono stati rappresentati anche i dati della serie 220k. E' stato disegnato un punto ogni 10 per comodità di visualizzazione. \label{fig: sck_log}}
\end{figure}

Per gli script si rimanda alla 
\href{https://github.com/LucaCiucci/relaz_seme/tree/master/Cartella_fit}{cartella}
, dove \verb+run.py+ esegue la corretta sequenza e \verb+config.py+ definisce 
i parametri fondamentali.

%================================================================
%                          Conclusioni
%================================================================
\section{Conclusioni}
Sulla base dell'analisi previamente effettuata, possiamo osservare che il
$\chi^2$ risulta essere minore rispetto al suo valore atteso. Ciò \`e
parzialmente dovuto alla correlazione degli errori inerenti i dati rilevati
all'interno dei due canali d'ingresso. In aggiunta, come si evince anche dalla
zona di maggior concentrazione dei residui all'interno dell'omonimo grafico,
il convertitore analogico-digitale del MCU \`e caratterizzato da una funzione
di risposta non lineare. Tale fattore era stato tenuto in considerazione
all'interno della calibrazione delle porte \verb+ADC0+ e \verb+ADC1+ mediante
l'aggiunta in quadratura di un errore di natura non statistica a quelli stimati
attraverso il fit lineare. Tale operazione avrebbe portato, quindi, ad
un'ulteriore diminuzione del $\chi^2$ all'interno del fit finale.
Ulteriori osservazioni possono essere condotte riguardo alle evidenti analogie
di carattere qualitativo fra la zona di maggior concentrazione dei punti e la
funzione di best fit.
In primo luogo, all'interno del grafico in scala lineare, a correnti alte
i punti risultano essere disposti pressoch\`e linearmente, in accordo con
l'ipotesi sulla componente resistiva interna. Nel grafico in scala
semilogaritmica, inoltre, i dati risultano possedere un andamento simile a
quello caratteristico della curva di Shockley, ovvero approssimativamente
rettilineo per poi appiattirsi come un logaritmo al crescere della tensione.
Dunque, possiamo asserire che l'andamento dei dati sperimentali risulta essere
in buon accordo con la legge \eqref{eq: model}.
Pertanto potremo aspettarci che i parametri stimati dal fit siano significativi 
e che la modellizzazione proposta sia una buona approssimazione del diodo reale 
nelle condizioni di lavoro considerate. La modellizzazione proposta può
risultare utile nelle simulazioni numeriche di circuiti con diodi in regimi
impulsati su alte correnti, dove sia richiesto un dettaglio sulle cadute di
tensione sui singoli componenti. Per tale scopo la legge \eqref{eq: model} \`e
confermata essere una buona approssimazione del diodo reale entro i limiti
sperimentali. Sebbene questo studio sia stato performato su un unico diodo
1N4007, il metodo proposto pu\`o essere esteso per qualunque altro tipo di
diodo, eventualmente sostituendo il condensatore con un opportuno circuito
stabilizzatore di corrente o di tensione.

%================================================================
%                        Appendice A
%================================================================
\section{Appendice A: Filtraggio Dati}\label{app: A}

%================================
%         Introduzione
%================================
\subsection{Introduzione}
All'interno dell'acquisione, sono stati raccolti un numero ingente di dati,
suddivisibili in base alla resistenza adottata e dunque facenti riferimento
a zone differenti della curva. A seguito della calibrazione, ci si è quindi
posto il problema di effettuare l'eliminazione degli outliers in modo
indipendente dalla scelta del modello per il fit. Le serie effettuate
variando la resistenza, inoltre, si sovrappongono in alcune zone del
grafico. Dunque è stato necessario eliminare i dati che, non aggiungendo
informazioni utili, andavano a "sporcare" il grafico.
Il sistema di filtraggio di dati implementato
nell'eseguibile si compone di 2 fasi: la prima consiste nell’eliminazione degli
outliers, la seconda dei dati non significativi.
%================================
%         Procedimento
%================================
\subsection{Procedimento}
Supponiamo di avere una serie di dati $(x, y)$ e assumiamo che siano
indipendenti tra loro. Quest'ipotesi non è vera in generale, ma
è tanto più lecita quanto più la correlazione tra le varianze delle misure su
$x$ e $y$ è indipendente dai valori assunti dalle $x$ e $y$ stesse e quanto più
sono numerosi i dati racchiusi entro una deviazione standard lungo $x$ per 
ciascun elemento: in questo caso, infatti, la correlazione viene inclusa nella
varianza lungo $y$.
Supponiamo inoltre che siano note a priori le $\sigma_x ^2 \coloneqq \var{x}$
e che la loro distribuzione di probabilità sia normale (le distribuzioni
delle componenti sono approssimativamente gaussiane per il convertitore
di \verb+Teensy+, perlomeno utilizzando la risoluzione a 12 bit) secondo una 
matrice
di covarianza diagonale nella base $\left\{x, y\right\}$.
In ogni modo, i nostri dati $x$ e $y$ risultano indipendenti e
approssimativamente normali. Dunque le assunzioni risultano giustificate. 
Conseguentemente la densità di probabilità che un punto misurato in $x$ si trovi
a tale ascissa $x_i$, si ricava integrando lungo $y$ a $x$ fissata:
\[
	\ud P = \frac{1}{\sigma_{x_i} \sqrt{2\pi}}
	e^{-\frac{1}{2}{\frac{(x - x_i)^2}{\sigma_{x_i}^2}}} \ud x
.\] 
Dunque, ripetendo più volte la stessa misura, si otterrà la probabilità:
\[
	P\left(\mid x - x_i \mid \leq \frac{\eps}{2} \right) = \eps G_{x_i} 
.\]
dove \[
	G_{x_i} \coloneqq \frac{1}{\sigma_{x_i} \sqrt{2\pi}}
	e^{-\frac{1}{2}{\frac{(x - x_i)^2}{\sigma_{x_i}^2}}}
.\] 
e $\eps > 0$ e $\eps \longrightarrow 0$. Scegliendo allora solo quelle misure $x$ per
cui vale $\mid x - x_i \mid \leq \frac{\eps}{2}$, queste saranno in numero
tendente a:
\[
	N_i \coloneqq N_{\text{tot}} \frac{G_{x_i}}{\sum_j G_{x_j}} =
		N_{\text{tot}} w_i
.\] 
che definisce implicitamente i pesi $w_i$ con cui si mediano le distribuzioni
di probabilità gaussiane $G_{x_i}$.
Allora, posto:
\[
	G_{y_i} \coloneqq \frac{1}{\sigma_{y_i} \sqrt{2\pi}}
	e^{-\frac{1}{2}{\frac{(\mu_y - y_i)^2}{\sigma_{y_i}^2}}}
.\] 
Per il principio di massima verosimiglianza siamo quindi interessati a
massimizzare la quantità:
\[
	\like = \prod_{i=1}^{n} \prod_{j=1}^{N_i} G_{y_i} = 
	\prod_{i=1}^{n} G_{y_i}^{N_i}
.\] 
Per la monotonia del logaritmo il problema equivale a massimizzare la quantità:
\[
	\ln{\like} = \sum_{i=1}^{n}\ln{G_{y_i}}^{N_{\text{tot}}w_i} = 
	\frac{N_{\text{tot}}} {\sum_{j=1}^{n} G_{x_j}} 
	\sum_{i=1}^{n} G_{x_i} \ln{G_{y_i}}
.\] 
Per cui, a meno di costanti risulta:
\begin{equation}\label{eq: likeconst}
	\ln{\like} - \text{const.} \propto \sum_{i=1}^{n} -G_{x_i} \ln{\sigma_y}
	- \frac{1}{2} G_{x_i} \left( \frac{y_i - \mu_y}{\sigma_y} \right)^2
\end{equation}
Imponendo la condizione di stazionarietà rispetto a $\mu_y$ si ottiene dunque:
\begin{equation}\label{eq: muy}
	\mu_y = \sum_{i=1}^{n} y_i w_i 
\end{equation} 
Una volta sostituito in \eqref{eq: likeconst} quanto appena trovato per $\mu_y$
e imponendo la stessa condizione di stazionarietà rispetto a $\sigma_y$ si ha:
\begin{equation}\label{eq: sigmay}
	\sigma_y^2 = \sum_{i=1}^{n} (y_i - \mu_y)^2 w_i
\end{equation}
Infine è possibile ricavare la varianza di $\mu_y$ dalla definizione di valore
di aspettazione, riconducendola più volte a integrali di gaussiane di altezze
e ampiezze diverse:
\begin{align} \label{aln: varmuy}
%	\var{\mu_y} &= \sum_{i=1}^{n} w_i^2 \sigma_y^2 + 
%	\left(\frac{y_i}{\sum_{j=1}^{n} w_j}  \right)^2 \frac{
%	\frac{e^{-\frac{(x-x_i)^2}{3 \sigma_x^2}}} {\sigma_x \sqrt{6 \pi} } +
%	\frac{e^{-3\frac{(x-x_i)^2}{4 \sigma_x^2}}} {\sigma_x \sqrt{\pi} } +
%	\frac{e^{-\frac{(x-x_i)^2}{\sigma_x^2}}} {\sigma_x \sqrt{2 \pi}}
%	} {\sqrt{2 \pi}} =\\ 
	\var{\mu_y} &= \sum_{i=1}^{n} w_i^2 \sigma_y^2 + 
	\left(\frac{y_i}{\sum_{j=1}^{n} w_j} \right)^2 \frac{
	e^{-\frac{(x-x_i)^2}{3 \sigma_{x_i}^2}} +  
	\sqrt{3} \left( e^{-\frac{(x-x_i)^2}{\sigma_{x_i}^2}} -
	\sqrt{2} e^{-3 \frac{(x-x_i)^2}{4 \sigma_{x_i}^2}} \right)
	} {2 \sqrt{3}\pi \sigma_{x_i}^2} 
\end{align}
Riassumendo:\\
Nella \eqref{eq: muy} prendiamo una media dei campionamenti intorno ad un'
ascissa $x$ in esame, pesata sulla distanza che gli $x_i$ hanno da questa; 
intuitivamente lo interpretiamo come se stessimo applicando un 
\emph{blur a kernel gaussiano} ai punti acquisiti.
Effettivamente quello che stiamo facendo non è molto diverso da KDE monovariante, dove però scaliamo secondo il valore delle $y$.
Lo stesso ragionamento vale per $\sigma_y^2$, si ha una stima della varianza
dei dati la variare di $y$, pesata sulla distanza dai valori studiati. Dunque
$\mu_y \pm \sigma_y$ ci dà una descrizione della distribuzione dei nostri dati.

%================================
%           Var(muy)
%================================
\subsection{$\var{\mu_y}$}
Mentre $\sigma_y$ rappresenta la distribuzione dei dati intorno al valor medio 
$\mu_y$, $\var{\mu_y}$ dà un'idea dell’incertezza che attribuiamo alla
miglior stima di $y$. Questo è utile per determinare la convergenza della
stima in funzione dei dati acquisiti.
Infatti tanto più è elevata la densità dei dati rispetto alla deviazione standard
$\sigma_x$, tanto più la stima del valore centrale risulta precisa. Graficamente
la banda di confidenza è più ristretta dove si concentrano i dati. Viceversa
tende ad allargarsi dove i dati sono sparsi, a distanze paragonabili a
$\sigma_x$. Numericamente, si vede dalla seconda somma nell'espressione
\eqref{aln: varmuy} che la stima del valore centrale è statisticamente
significativa solo quando si media su un intervallo campionato con almeno
qualche punto ogni deviazione $\sigma_x$: altrimenti $\sigma_y \to 0$
indicando così assenza di dati, mentre $\var{\mu_y}$ tende a $+\infty$ come
$\sim e^{x^2}$, indice della stessa insufficienza di dati al fine di stabilire
con precisione significativa il valore di $\mu_y$.
\begin{figure}[!htbp]
	\centering 
 		\includegraphics[scale=0.32]{./varmuy.png}
 	\caption{La media $\mu_y$ è rappresentata dalla linea rossa, mentre
	l'area in rosso indica il valore di $\var{\mu_y}$ al variare dei
	dati (in blu) lungo $x$. \label{fig: varmuy}}
\end{figure}
Nel caso opposto, in cui i dati sono "densi" (in confronto alle $\sigma_x$)
la seconda somma, per quanto computazionalmente intensiva, numericamente
sembrerebbe piccola in confronto alla prima: in realtà non lo è, ma
soprattutto questa non può essere trascurata, poiché è proprio la quantità
che descrive la dipendenza dalla densità stessa e dunque la caratteristica
convergenza/divergenza della precisione sulla stima centrale fornita.

%================================
%        Filtro outliers
%================================
\subsection{Filtro outliers}
La parte più semplice nel filtraggio dati consiste nello scartare tutti quei
punti che distano da $\mu_y$ più di una soglia arbitraria $k$ di deviazioni
standard $\sigma_y$ (nel nostro caso è stato scelto $k = 2$, non critico,
trovato dopo una serie di prove). A differenza del classico metodo basato
sulla distanza della curva modello di best fit, non siamo influenzati da
quest’ultimo. Questo risulta particolarmente utile in simili situazioni di
verifica del modello in quanto una selezione basata su un preliminare fit
risulterebbe influenzata dalla scelta della funzione in questione e
eliminerebbe tutti i dati che non risultano compatibili con essa.
%indipendentemente dal modello di fit vale $|y_i - \mu_y| \geq k\sigma_y$.

%================================
%  Filtro dati non significativi
%================================
\subsection{Filtro dati non significativi}
Supponiamo di avere 2 set di dati fatti con diverse resistenze, il primo $(A)$
con una resistenza bassa, il secondo $(B)$ con una alta: Il primo set esplorerà
la regione ad alta corrente, mentre il secondo la regione di basse correnti.
In generale i dati del primo si sovrapporranno anche nelle zone basse esplorate
dal secondo, però senza aggiungere sostanziali informazioni rispetto a quanto
farebbe il secondo.
%Il nostro obiettivo è dunque eliminare questi dati meno significativi.
Esponiamo dunque il criterio sviluppato per ridurre l'influenza di questi
punti meno significativi sulla ricerca dei parametri di best-fit e sulla
rappresentazione finale dei dati.\\
Per capire se in un certo punto i dati di $A$ sono significativi, calcoliamo
la misura di significatività che abbiamo sviluppato in \eqref{aln: varmuy}$:
\var{\mu_y}$ di $A$ e di $B$. Perciò se $\var{\mu_y}$ di $A$ è maggiore di 
$q\var{\mu_y}$ di $B$, con $q$ arbitrario (nell’esperienza è stato scelto
$q = 3$), questo indica che i dati di $A$
ci stanno dando "poca" informazione rispetto a quelli di $B$. A questo punto
è sufficiente controllare tutti i punti scorrendo su tutte le combinazioni
di set per eliminare i dati non significativi, che rendono meno
immediata l'interpretazione del grafico. Questo è ben visibile in scala
logaritmica sulle $y$ dove i punti con grandi incertezze o varianze tendono
a disperdersi rapidamente.
L’algoritmo è computazionalmente intensivo e richiede una corretta gestione
della memoria per evitare bolle di allocazione, dunque è stato implementato
in \verb'C++' per praticità e richiamato all’interno degli script, per dettagli si 
rimanda ai 
\href{https://github.com/LucaCiucci/relaz_seme/tree/master/Cartella_fit/filter_src}{sorgenti}.
Nelle figure di esempio sono mostrati i dati selezionati dall’algoritmo
(in nero) ed i dati scartati (in rosso). 
E’ infine mostrato il confronto dei grafici delle $\var{\mu_y}$
tra due set successivi.

\begin{figure}[!htbp]% per questi non ci dovrebbero essere problemi, basta che non vanno in B
\centering
\begin{subfigure}{.5\textwidth}
	\centering 
 		\includegraphics[scale=0.5]{./nofilter.png}
%	\caption{Grafico in scala semilogaritmica prima del filtraggio dati:
%	si evidenziano in rosso i dati da scartare, in nero quello tenuti.}
	\label{fig: nofilter}
\end{subfigure}%
\begin{subfigure}{.5\textwidth}
	\centering 
 		\includegraphics[scale=0.5]{./filtered.png}
% 	\caption{Grafico in scala semilogaritmica dopo il filtraggio dati.}
	\label{fig: filtered}
\end{subfigure}
	\caption{Grafici in scala semilogaritmica dei prima (sinistra) e dopo 
(destra) del filtraggio dati. I dati scartati sono stati evidenziati in rosso. 
Per praticità è stato rappresentato un centesimo dei dati raccolti}
\end{figure}
\begin{figure}[!htbp]
	\centering 
 		\includegraphics[scale=0.32]{./comparison.png}
 	\caption{Confronto dei grafici delle $\var{\mu_y}$ su due set di
	dati consecutivi. \label{fig: comparison}}
\end{figure}

%================================================================
%                        Appendice B
%================================================================
\section{Appendice B: Metodo di fit}\label{app: B}
La differenza di potenziale $\Delta$V in funzione della corrente $I$ che scorre 
all’interno del diodo può essere espressa attraverso la legge
\eqref{eq: model}. Potremo, dunque, essere tentati ad adottare tale formula
ai fini di un fit numerico. 
Tuttavia tale operazione provocherebbe il fallimento del fit a causa dei valori 
negativi o nulli che debitamente si riscontreranno all’interno del logaritmo. 
E’ stato dunque necessario adottare un algoritmo alternativo, basato sul 
metodo di Newton\cite{tesi}. Esso ci consentirà di ricostruire numericamente 
la funzione inversa della legge \eqref{eq: model}.
Illustriamo in breve il metodo di Newton. Sia data una funzione $f(x)$ continua e 
derivabile entro un intervallo di definizione connesso (con $f’(x)$ limitata e 
diversa da $0$ per ciascuna $x$ appartenente al dominio) ed una $v$ tale che
$f(v) = 0$. Sviluppando al prim’ordine f in un intorno di v, otterremo:
\begin{equation}
f(x) = f(v) + f{'}(v) + {\frac{1}{2}} {f{''}(X)} \cdot  {( x - v)^2} = 0
\end{equation}
ove $X$ è un numero reale compreso fra $v$ ed $x$. Nel caso in cui il $(x-v)^2 
$risultasse prossimo a zero:
\begin{equation}
x \simeq v   -  {\frac {f(v)}{f’(v)}}
\end{equation}
L’unica limitazione nell’adottare tale scrittura è che necessitiamo di $v$ 
molto vicino a $x$ incognito. Tuttavia, nel caso in cui la funzione $f$ risultasse 
sufficientemente maneggevole, potremo partire da un valore generico del dominio 
e riscontrare con buona approssimazione la radice in un tempo finito 
relativamente breve. Per effettuare ciò, è necessaria la serie ricorsiva:
\begin{equation}
\begin{cases}
x[0] = a \\ x[N+1] = x[N] - \frac {f(x[N])} {f’(x[N])}
\end{cases}
\end{equation}
ove $a$ è un generico valore appartenente al dominio di definizione. In pratica, 
per ogni iterazione tale serie valuta $f(x[i])$ e cerca la retta tangente in tale 
punto alla funzione:
\begin{equation}
y = f(x[i]) + f’(x[i]) \cdot (x-x[i])
\end{equation}
 Dopodiché cerca il punto $x$ per il quale $y = 0$:
\begin{equation}
x = x[i+1] = x[i] - \frac{f(x[i])}{f’([i])}
\end{equation}
E viene conseguentemente valutato il valore $f(x[i+1])$ ed il procedimento viene 
ripetuto per un dato numero di iterazioni. 
\begin{figure}[!htbp]
	\centering 
 		\includegraphics[scale=0.75]{./Figura1_appendiceB.png}
	\caption{Visualizzazione grafica di cosa implica un’iterazione della 
serie ricorsiva (f(x) = $x^3$ in questo esempio)}
\end{figure}
Se la funzione è monotona e la serie converge ad un valore, tale sarà $v$, 
essendo l’unico punto fisso della serie:
\begin{equation}
 {a = a -  {\frac{f(a)}{f{’}(a)}}}\to{{f(a) = 0}\land{v = a}}
\end{equation}
essendo $v$ l’unico punto in cui si annulla la funzione per definizione.
Ritornando al diodo, esso può essere interpretato quale un diodo ideale ed una 
resistenza in serie, collegati ad un generatore di differenza di potenziale. 
Quindi, per la legge delle maglie:
\begin{equation}
\Delta V_{\text{ingresso}} = \Delta V_{\text{diodo}} + \Delta 
V_{\text{resistenza}} = \Delta V_{\text{diodo}} + R \cdot I
\end{equation}
ove $I$ è la corrente che scorre all’interno del diodo in quel determinato 
istante. Notoriamente, tale scrittura si può porre nella seguente forma ai 
fini di definire la retta di carico del diodo:
\begin{equation}
\Delta V_{\text{diodo}} = \Delta V_{\text{ingresso}} - R \cdot I
\end{equation}
Il punto di intersezione di tale retta con l’equazione di Shockley per il 
diodo rappresenta il punto di lavoro $(v, i)$ del diodo:
\begin{equation}
i = I_0 \cdot ( e^{\frac {\Delta V_{\text{ingresso}} - R \cdot i}{\eta V_T} } - 
1) = I_0 \cdot (e^{\frac{v} {\eta V_T}} -1) =  \frac{\Delta V_{\text{ingresso}} 
- v}{R}
\end{equation}
Tale i dovrà necessariamente corrispondere alla corrente di lavoro in virtù 
dell’equazione di Shockley. Dunque definendo la funzione $g(x)$ come segue:
\begin{equation}\label{eq: invsck}
g(x) = I_0 \cdot ( e^{\frac{x}{\eta V_T}} - 1)  - \frac {\Delta 
V_{\text{ingresso}} - x}{R}
\end{equation}
è possibile adottare il metodo di Newton al fine di riscontrare la radice $v$
ed individuare così la corrente di lavoro. 
Dunque si potrà effettuare un fit della corrente in funzione della tensione ai 
capi del diodo reale attraverso una legge che per un determinato valore della 
tensione ($\Delta V_{\text{ingresso}}$) ed un dato numero di iterazioni (ne 
sono sufficienti poche nel nostro caso ai fini di ottenere un’approssimazione 
adeguata), riesce ad individuare la radice $v$ nelle modalità previamente 
descritte. Attraverso la stessa, sarà quindi possibile individuare la corrente 
di lavoro del diodo per la suddetta $\Delta V_{\text{ingresso}}$. 
\`E immediato comprendere che fare un fit con questa legge risulta essere 
equivalente a fare un fit con la legge \eqref{eq: model}, con minori complicazioni 
computazionali. Verifichiamo graficamente che per dei generici valori della 
corrente  i due metodi danno risultati sostanzialmente equivalenti (adottiamo i 
parametri stimati dal fit). I grafici relativi alla convergenza di $v$ e di $I$ 
anche a seguito di sole 20 iterazioni:
\begin{figure}[H]% questi vanno sistemati meglio perchè occupano una pagina da soli!!!
	\centering 
 		\includegraphics[scale=0.75]{./Figura2_appendiceB.png}
\end{figure}
\begin{figure}[H]
	\centering 
 		\includegraphics[scale=0.75]{./Figura3_appendiceB.png}
\end{figure}

Tali grafici evidenziano che, come era da aspettarsi, la legge adottata per il 
fit risulta essere l’inversa numerica della legge \eqref{eq: model}. La serie, inoltre, 
risulta convergere molto rapidamente ai valori attesi e quindi sono 
effettivamente sufficienti solo 20 iterazioni per una buona precisione.
All’interno della funzione di fit, è stato infine introdotto un offset quale parametro
libero per ovviare errori di zero dovuti agli assorbimenti di ADC in Teensy 3.2.


%================================================================
%                            END
%================================================================
\medskip
\bibliographystyle{IEEEtrandoi}
\bibliography{refs}
\end{document}
