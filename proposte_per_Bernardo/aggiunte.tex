\documentclass[11pt, a4paper] {article}
\title{Modifiche}
\date{16 settembre 2019}
\usepackage{graphicx}
\usepackage{verbatim}
\begin{document}
\maketitle
\section{Cenni Teorici}
Il diodo e la resistenza $R_1$ costituiscono nel loro insieme un partitore di tensione. Pertanto la corrente che scorre all'interno dei medesimi è la stessa. Un diodo reale può essere a sua volta schematizzato quale un ramo costituito da un resistore ohmico ed un diodo ideale in serie. Pertanto ci aspettiamo che la relazione che lega la corrente e la tensione ai capi del diodo possa essere espresso secondo la legge:
\begin{equation}
\Delta V = \Delta V_{diodo} + \Delta V_{resistore} = \eta V_T  \ln{\frac{I+I_0}{I_0}} + RI
\end{equation}
\section{Analisi dati}
Il nostro intento è verificare che i dati raccolti siano in accordo con la legge 1. Per la legge di Ohm è possibile calcolare la corrente nel diodo, dividendo le tensioni rilevate all'interno della boccola ADC0 per la resistenza R. E'  stato quindi effettuato un filtraggio volto all'eliminazione degli offset e dei punti meno significativi, assumendoli quali variabili indipendenti e di natura gaussiana. Per maggiori dettagli, si rimanda alla sezione A delle Appendici.
Successivamente, è stato effettuato un fit sulla base dei dati selezionati. Ingenuamente, avremo dunque potuto pensare di adottare la legge 1 direttamente. Tuttavia ciò comportrebbe il fallimento repentino del fit a causa dei valori negativi o nulli che debitamente si riscontrerebbero entro il logaritmo. E' stata quindi adottata una legge alternativa basata sul metodo delle tangenti (o metodo di Newton). Per maggiori dettagli, si rimanda alla sezione B delle Appendici. I dati raccolti e la funzione di fit sono stati posti all'interno del grafico 1.
[inserire grafico 1]
I parametri stimati dal fit risultano essere:
\begin{equation}
R_{diodo}= 0.0475 \pm 0.0002  \Omega
\end{equation}
\begin{equation}
\eta V_T = 0.04640 \pm 0.00006 V
\end{equation}
\begin{equation}
I_0 = (3.18 \pm 0.05) \cdot 10^{-9} A
\end{equation}
Il $\chi^2$ risulta essere pari a 4483 contro un aspettato di 2760. (in dati vanno cambiati con quelli del fit fatto bene). Successivamente, i dati e la funzione sono stati posti all'interno di un grafico in scala bilogaritmica (grafico 2).
[inserire grafico 2]

\section{Conclusioni}
L'andamento dei dati sperimentali non risulta essere bene modellizzato dalla legge 1. Ciò può essere spiegato sulla base della non linearità dei convertitori analogici digitali all'interno di Teensy 3.2, che ha comportato degli errori difficilmente stimabili e presumibilmente correlati. Tuttavia, si riscontrano delle analogie tra la funzione di fit ed i dati ottenuti. In particolare, dal grafico in scala lineare, si osserva che a correnti alte l'andamento risulta essere pressochè lineare in accordo con l'ipotesi sulla resistenza interna. Dal grafico in scala bilogaritmica, inoltre, osserviamo che i dati risultano possedere un andamento simile a quello carateristico della curva di Shockley per poi appiattirsi a partire da un particolare valore della tensione, esattamente come ci aspetteremo da un andamento lineare. Pertanto potremo aspettarci che i parametri stimati dal fit siano significativi.

\end{document}
